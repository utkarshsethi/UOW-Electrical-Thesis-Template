\chapter{Another Chapter}\label{chap:another}

This is Chapter number~\ref{chap:another}. This is where
things start to get very dull.

Go on and describe contents of chapter, blah blah blah \ldots


\section{First Section in Chapter}\label{another:sec}

This is section~\ref{another:sec}.

\section{Another Section in Another Chapter}

We're very keen so we are referencing Table~\ref{conversion}

\begin{table}[!h]
\centering
\begin{tabular}{l|l|l|l|}
\multicolumn{1}{l}{}&\multicolumn{1}{l}{}&\multicolumn{2}{c}{TO}\\ 
\cline{3-4}
\multicolumn{1}{l}{}&\multicolumn{1}{c|}{}&\multicolumn{1}{c|}{AC}&\multicolumn{1}{c|}{DC}\\ 
\cline{2-4}
\multirow{2}{*}{FROM}&\multicolumn{1}{c|}{AC}&Cycloconverter&Rectifier\\ 
&\multicolumn{1}{c|}{DC}&Inverter&Chopper\\ 
\cline{2-4}
\end{tabular}
\caption[Caption for List of Tables]{Classification of Conversion Circuits}
\label{conversion}
\end{table}

\section{Complicated Equations}

One thing that \LaTeX is really good at is typesetting mathematics.
\begin{equation}
\frac{d^2 \Phi_q(\theta)}{d\theta^2}-\frac{2\mu_o R^2 L_R}{p^2 g_e}\Phi_q(\theta)
+\frac{\mu_o R^2 L_R}{p^2 g_e} \left[J(\theta)-J(\pi-\theta)\right]=0
\label{aeqn:mm16}
\end{equation}
Arrays are used for really long equations.
\begin{eqnarray}
2RJ_q=2A\left[\frac{\Re_qR}{\gamma}(e^{\gamma\frac{\theta_p}{2}}-e^{-\gamma\frac{\theta_p}{2}})
+p\Re_{side}(e^{\gamma\frac{\theta_p}{2}}+e^{-\gamma\frac{\theta_p}{2}})\right]+
\nonumber\\
\frac{4cJ_q}{b+1}\left[\Re_qR\sin\frac{\theta_p}{2}+p\Re_{side}\cos\frac{\theta_p}{2}\right]
\end{eqnarray}


\section{Summary}

Blah blah \ldots
